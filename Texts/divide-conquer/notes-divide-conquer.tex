\documentclass{article}
\usepackage{authblk}
\usepackage{listings}


\begin{document}

\title{Divide and Conquer}
\author{Vitor R. Greati\thanks{greati@ufrn.edu.br || vitorgreati.me}}
\date{December, 2015}

\maketitle

The divide-and-conquer approach usually falls on recursive solutions, and is based on three main
steps:
\begin{description}
	\item[Divide] Divide the problem into smaller instances of the same problem.
	\item[Conquer] Solve each minor instance using a straightfoward manner.
	\item[Combine] Combine those solutions to finally solve the original problem.
\end{description}

Given the recursive aspect, \textbf{recurrences} are the most common ways to define functions for
counting the total of operations performed by the solutions. They simply use smaller inputs to describe
a function, which means that there is not a direct definition, but a description based on known outputs
of that same function.

Three methods can be applied for solving recurrences:
\begin{itemize}
	\item{Substitution method}
	\item{Recursion-tree method}
	\item{Master method}
\end{itemize}

Sometimes, details are ignored when formulating recurrences. For example, on Merge Sort algorithm's,
there are floor and ceiling functions involved, but they turn out to disappear when theta notation
is applied. Otherwise, there are some cases where this simplification can't occur.

\end{document}
