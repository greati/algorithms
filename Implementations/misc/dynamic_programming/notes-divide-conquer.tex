\documentclass{article}
\usepackage{authblk}
\usepackage{listings}


\begin{document}

\title{Dynamic Programming}
\author{Vitor R. Greati\thanks{greati@ufrn.edu.br || vitorgreati.me}}
\date{January, 2016}

\maketitle

\section{TopCoder explanations}

We have the following problem:

\emph{Given a list of $N$ coins, their values $\(V_1, V_2, \ldots, V_N)$, and the total sum $S$. Find the minimum number of coins the sum of which is $S$ (we can use as many coins of one type as we want), or report that it’s not possible to select coins in such a way that they sum up to $S$.}

\subsection Finding a state

A state is a way to describe a situation representing a sub-solution for the problem. For example,
if we want to find a solution for sum $i \le S$, this would be a state for $i$. A smaller state
than $i$ would be a solution for any $j < i$. For finding a state for $i$, it's necessary to
find all states for all $j < i$. Once found for $i$, it's easy to find for $i + 1$, until
$i + 1 = S$.




\end{document}
